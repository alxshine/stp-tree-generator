\documentclass{beamer}
\usepackage{uibkstyle}
\usepackage[utf8]{inputenc}
\usepackage[german]{babel}
\usepackage{tikz}

\tikzset{onslide/.code args={<#1>#2}{%
  \only<#1>{\pgfkeysalso{#2}} 
}}
\tikzset{
    other/.style={circle, onslide=<1-5>{white}}
}

\title{STPViz}
\subtitle{Visualizing network topologies with the help of the Spanning Tree Protocol}
\author{Alexander Schlögl}

\begin{document}
\begin{frame}[plain]
    \maketitle
\end{frame}

\begin{frame}{Überblick}
    \begin{itemize}
        \item \textbf{Einleitung \& Motivation}
        \item \textbf{Zeitplanung}
        \item \textbf{Spanning Tree Protocol (STP)}
        \item \textbf{STPViz}
        \item \textbf{Software-Switch}
        \item \textbf{Tests}
        \item \textbf{Zusammenfassung \& Ausblick}
    \end{itemize}
\end{frame}

\begin{frame}{Einleitung \& Motivation}
    \begin{itemize}
        \item Warum STP?
        \item Was ist das Problem?
        \item Was macht STPViz besser/einfacher?
    \end{itemize}
\end{frame}

\begin{frame}{Timeline}
    \begin{itemize}
        \item Geplante Timeline
        \item Echte Timeline (mit Problemen)
    \end{itemize}
\end{frame}

\begin{frame}{STP}
    \begin{itemize}
        \item Funktionsweise
        \item Pakete
    \end{itemize}
\end{frame}

\begin{frame}{STPViz}
    \begin{itemize}
        \item Struktur \& Funktionsweise
        \item Probleme
        \item Fehlerkorrektur
        \item Darstellung
    \end{itemize}
\end{frame}

\begin{frame}{Grafik}
    \centering
    \begin{tikzpicture}[nodes=draw]
        \node[circle, onslide=<1>{red}, onslide=<2-4>{white},onslide=<5-6>{red}] (r) at (16,10) {};

        \node[circle,onslide=<1-3>{white},onslide=<4>{red}] (a0) at (13, 9) {};
        \node[other] (a1) at (15.5, 9) {};
        \node[other] (a2) at (18, 9) {};

        \node[other] (b0) at (12, 8) {};
        \node[circle,onslide=<1-2>{white},onslide=<3>{red}] (b1) at (13.7, 8) {};
        \node[other] (b2) at (14.5, 8) {};
        \node[other] (b3) at (16, 8) {};
        \node[other] (b4) at (17, 8) {};
        \node[other] (b5) at (19, 8) {};

        \node[circle,onslide=<1>{white},onslide=<2>{red}] (c0) at (13, 7) {};
        \node[other] (c1) at (15, 7) {};
        \node[other] (c2) at (17, 7) {};
        \node[other] (c3) at (18, 7) {};

        \node[circle] (d0) at (13.5, 6) {};
        \node[other] (d1) at (15.5, 6) {};
        \node[other] (d2) at (17, 6) {};

        \draw[onslide=<1>{white}]
        (d0) -- (c0);

        \draw[onslide=<1-2>{white}]
        (c0) -- (b1);

        \draw[onslide=<1-3>{white}]
        (b1) -- (a0);

        \draw[onslide=<1-4>{white}]
        (a0) -- (r);

        \draw[other]
        %stp links
        (r) -- (a1)
        (r) -- (a2)
        
        (a0) -- (b0)
        (a0) -- (b2)

        (a1) -- (b3)

        (a2) -- (b4)
        (a2) -- (b5)

        (b3) -- (c1)

        (b4) -- (c2)

        (b5) -- (c3)

        (c2) -- (d1)

        (c3) -- (d2);

        \draw[white]
        %other links
        (a0) -- (b3)
        (a2) -- (b3)
        (b0) -- (c0)
        (b2) -- (c0)
        (b3) -- (c2)
        (c2) -- (c3)
        (c1) -- (d1)
        (b2) -- (c1)
        (c1) -- (d0);
    \end{tikzpicture}
\end{frame}

\begin{frame}{Software-Switch}
    \begin{itemize}
        \item Grund
        \item Funktionsweise
        \item Grenzen \& Beschränkungen
    \end{itemize}
\end{frame}

\begin{frame}{Testing}
    \begin{itemize}
        \item Setup
        \item Physisches Setup
        \item Tests
        \item Resultate
    \end{itemize}
\end{frame}

\begin{frame}{Zusammenfassung und Ausblick}
    \begin{itemize}
        \item STPViz Fähigkeiten
        \item STPViz Grenzen
        \item Software-Switch
        \item Sonstige Dinge (OpenWrt, dd-wrt)
    \end{itemize}
\end{frame}

\begin{frame}{Message}
Ich hätte gerne, dass Zuseher folgendes mit nach Hause nehmen:\\
\begin{itemize}
    \item Information aus STP zu extrahieren ist schwer, da es nur lokales Wissen benutzt.
    \item Wir haben es trotzdem geschafft (nur halt nicht mit maximaler Informationsdichte).
    \item Es gibt nicht viele Use Cases, aber es gibt sie.
    \item STPViz ist eine gute Grundlage für weitere Arbeiten.
\end{itemize}
\end{frame}

\end{document}
