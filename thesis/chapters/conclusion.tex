\chapter{Conclusion}
\label{conclusion}
The purpose of this thesis was to create a distributed, passive, low impact network visualization tool that relies solely on STP packets.
With \tool, we created a network visualization tool that does just that.
It can extract information about bridges from STP packets, and under the right circumstances even make assumptions about bridge connections.
When the established spanning tree is expanded, and a new root is prepended, \tool\ expands its known topology with it.
The inherent uncertainty of these assumptions is reduced by combining data obtained from distributed nodes.
This means that the usefulness of \tool\ increases greatly with the number of clients running on a network.
We tested \tool\ to the greatest extend possible with the limited hardware we had at our disposal.
In order to perform these tests we had to develop \textit{software-switch}, a utility that mimics STP capable bridges.
It is a very basic and minimal tool, developed just far enough to make it usable for our tests.
\tool\ passed all the tests we designed (in static and dynamically changing networks).
The tests showed that \tool\ is fully capable of bridge discovery and handling a bridge's removal, as well as making assumptions about bridge connections on topology updates.
In our tests, \tool\ correctly filtered out incorrect assumptions in the server by comparing information from different clients.
When confronted with sudden changes in the physical network topology, \tool\ proved to be stable enough to handle them without discarding any correct assumptions made beforehand.

Unfortunately we did not find any other general method of checking assumed topologies.
The lack of mechanics like a hop count on the Data Link layer keeps us from implementing something like a \textit{traceroute}, which we could use to check our assumptions manually.

Future extensions of \tool\ could include features to increase usability, like a GUI, as well as other output formats.
\tool\ could also be extended to include assumptions about actual cables by using the root path cost.
Extending the functionality of \tool\ to STP extension, as well as alternatives to STP (discussed in Section~\ref{stp}) would also be very helpful.
An even better extension would be to detect and use whichever features the connected bridge is capable of dynamically at runtime.
Additionally we should extend the parser to be capable of generating output from one or more \textit{.pcapng} files without needing to launch a server and multiple clients beforehand.
With more extensive testing and statistical analysis of the results, it is our opinion that the assumptions we make could be improved as well.
As we realized during the Slow Dynamic Change Test (Section~\ref{slow_dynamic_test}), TC flags are not always sent when they should be.
Any future revision of this tool should consider finding another method of detecting network changes, to increase accuracy.

The \textit{software-switch} utility could also be extended and tested more thoroughly to make it usable as a general purpose switching utility.
For easier testing (e.g. with \tool) it would be beneficial to find a way to create multiple virtual switches on a single device.
This would greatly increase testing speed and reduce the amount of hardware required for testing large networking scenarios. 

Other extensions of this thesis include work on the \textit{OpenWrt} and \textit{dd-wrt} projects in order to find a viable method to keep the STP implementation consistent with other hardware.
Informing their respective communities of this issue and working together to find a solution would be a good start.
Searching for ways to keep STP implementations consistent across devices and distributions would be a great benefit to the networking community.
