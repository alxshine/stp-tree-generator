\chapter{Conclusion}
\label{conclusion}
The purpose of this thesis was to create a distributed, passive, low impact network visualization tool that relies solely on STP packets.
With \tool, we created a network visualization tool that does just that.
It can extract information about bridges from STP packets, and under the right circumstances even make assumptions about bridge connections.
When the established spanning tree is expanded, and a new root is prepended, \tool\ expands its known topology with it.
The inherent uncertainty of these assumptions is reduced by combining data obtained from distributed nodes.
This means that the usefulness of \tool\ increases greatly with the number of clients running on a network.
Unfortunately we did not find any other general method of checking assumed topologies.
The lack of mechanics like a hop count on the Data Link layer keeps us from implementing something like a \textit{traceroute}, which we could use to check our assumptions manually.

Future extensions of \tool\ could include features to increase usability, like a GUI, as well as other output formats.
Extending the functionality of \tool\ to STP extension, as well as alternatives to STP (discussed in Section~\ref{stp}) would also be very helpful.
Additionally we should extend the parser to be capable of generating output from one or more \textit{.pcapng} files without needing to launch a server and multiple clients beforehand.
With more extensive testing and statistical analysis of the results, it is our opininon that the assumptions we make could be greatly improved.

The \textit{software-switch} utility could also be extended and tested more thoroughly, to make it usable as a general purpose switching utility.
To this end, the update of root information could be improved.

Other extensions of this thesis include work on the \textit{OpenWrt} and \textit{dd-wrt} projects in order to find a viable method to keep the STP implementation consistent with other hardware.
