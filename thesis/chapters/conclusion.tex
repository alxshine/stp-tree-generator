\chapter*{Conclusion}
Gaining information about bridges in the network is easy.
Gaining information about the connections in between is a lot harder, and only possible under certain conditions.
By observing changes to the packets sent during buildup of the spanning tree, we can the tree has a certain shape.
These assumptions, however, are not correct in every case.
We can only try to minimize these errors by combining data sent from multiple clients.
This means that the usefulness of the tool discussed increases greatly with the number of clients running on a network.
Unfortunately we did not find any other general method of checking assumed topologies.
The lack of mechanics like a hop count on the Data Link layer keeps us from implementing something like a \textit{traceroute}, which we could use to check our assumptions manually.

Future extensions of this tool could include features to increase usability, like a GUI, as well as other output formats.
The \textit{README} in the git should also be extended to include the parts of this thesis paper that are vital to using the tool.
Additionally we should extend the parser to be capable of generating output from one or more \textit{.pcapng} files without needing to launch a server and multiple clients beforehand.
\newline

The \textit{software-switch} utility could also be extended and tested, in order to make it usable as a general purpose switching utility.
Other extensions of this thesis include work on the \textit{OpenWrt} and \textit{dd-wrt} projects in order to find a viable method to keep the STP implementation consistent with other hardware.
%TODO:extend? compare to other papers
