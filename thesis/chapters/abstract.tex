\begin{abstractpage}
    \begin{abstract}{english}
        Monitoring the logical topology of Spanning Tree Protocol (STP) enabled networks is a difficult task, and often only accomplished by using other network management protocols.
        Not all networking hardware supports these protocols, making this a suboptimal approach.
        In this thesis we present \tool, a tool which can monitor nodes in a network using only passively obtained STP information.
        It uses a distributed client-server architecture to gather STP packets at multiple points in the network.
        STP transmits only local data, but \tool\ can still use this distributed information to draw conclusions about connections in the network.
        As our method of assuming connections in the network does not guarantee correctness, we provide a simple and computationally cheap way of eliminating errors.
        The error eliminating accuracy increases with the number of \tool\ instances in the network, going up to 100\% if there is an \tool\ instance connected to every networking device.
        Due to inconsistencies we found in STP implementations during testing, we had to implement the protocol ourselves to be able to test \tool.
        \tool\ passed all the tests we designed.
        It can handle node discovery, connection discovery, device removal and even dynamic topolgy changes.
        These test results leave us with the impression that with usability improvements \tool\ can become a useful addition to the network monitoring suite for environments where the networking hardware does not support more complex management protocols.
    \end{abstract}

    \begin{abstract}{german}
        Die Überprüfung von Netzwerktopologien, welche das Spanning Tree Protocol (STP) verwenden, ist eine schwere Aufgabe.
        Diese kann oft nur mit Hilfe von Netzwerkverwaltungsprotokollen bewältigt werden.
        Da nicht alle Netzwerkgeräte diese Protokolle unterstützen, ist dies eine suboptimale Lösung.
        In dieser Arbeit präsentieren wir \tool, ein Werkzeug, welches Netzwerktopologien nur anhand von passiv erhaltenen STP Paketen beobachten kann. 
        Es verwendet eine verteilte Client-Server-Architektur um STP Pakete an mehreren Orten im Netzwerk zu erhalten.
        Obwohl STP nur lokale Daten überträgt, kann \tool\ durch diese verteilten Informationen Vermutungen zu Verbindungen im Netzwerk anstellen.
        Da unsere Annahmen nicht immer korrekt sind, beschreiben wir auch eine einfache und leicht berechenbare Methode zur Entfernung dieser Fehler.
        Die Genauigkeit unserer Methode wächst mit den \tool\ Instanzen im Netzwerk, und erreicht 100\% wenn eine \tool\ Instanz mit jedem Netzwerkgerät verbunden ist.
        Durch Ungenauigkeiten in STP Implementationen, die wir beim Test von \tool\ fanden, waren wir gezwungen, STP selbst zu implementieren.
        Erst dadurch konnten wir unsere Tests durchführen, welche \tool\ alle bestand.
        Dabei wurden seine Fähigkeiten zur Entdeckung von Geräten, Annahmen zu Verbindungen zwischen Geräten, die Entfernung von Geräten sowie dynamische Topologieveränderungen getestet.
        Diese Testresultate verleiten uns zur Annahme, dass \tool\ mit verbesserter grafischer Ausgabe ein nützliches Werkzeug zur Netzwerkkontrolle in Umgebungen ohne Unterstützung von komplexeren Netzwerkverwaltungsprotokollen sein kann.
    \end{abstract}
\end{abstractpage}

\selectlanguage{english}
